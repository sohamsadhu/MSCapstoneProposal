\documentclass[12pt]{artikel3}                  % I need a font of 12 for the proposal.
\linespread{1.2}                                % Not sure why we need this here.
\usepackage{fullpage, setspace, graphicx}       % What does the full page library do?
\usepackage{mathtools, amsfonts}                % Mathtools are understandable, and so are the amsfonts that are being used.
\usepackage[margin=1in]{geometry}
\usepackage{newcent}                            % Why do I need the newcent package or what it does?
\usepackage{url}                                % If I am including URL in any place, I will need this package.
\usepackage{cite}                               % Need this for citing of the various formats.
\usepackage{hyperref}

\begin{document}
\begin{titlepage}
\begin{center}

\textsc{\LARGE }\\[0.2cm]
\textsc{\LARGE }\\[0.2cm]
\textsc{\LARGE }\\[1.2cm]

\textsc{\Large Master's Project Proposal}\\[1cm]

\large{Soham Sadhu}\\
\large{Department of Computer Science}\\
\large{Rochester Institute of Technology}\\
\large{Rochester, NY 14623}\\
\large{sxs9174@rit.edu}\\[0.5cm]

{\large \today}\\[1cm]

\begin{tabular}{l l r}
    Chair: & Prof. Stanis{\l}aw Radziszowski & spr@cs.rit.edu\\[1.5cm] \hline
    \multicolumn{3}{c}{signature \hspace{6cm} date}\\[1cm]
    Reader: & Prof. Alan Kaminsky & ark@cs.rit.edu\\[1.5cm] \hline
    \multicolumn{3}{c}{signature \hspace{6cm} date}\\[1cm]
    Observer: & Prof. Edith Hemaspaandra & eh@cs.rit.edu\\[1.5cm] \hline
    \multicolumn{3}{c}{signature \hspace{6cm} date}\\[1cm]
\end{tabular}


\vfill

\end{center}
\end{titlepage}

\begin{center}
\parbox{350pt}{
  \begin{center}\textsc{Abstract}\end{center}
  \vspace{0.5cm}
  Hash functions, have \href{"http://en.wikipedia.org/wiki/Cryptographic\_hash\_function\#Applications"}
  {applications in computer security}, in fields of authentication and integrity.
  Due to importance of hash function usage in everyday computing, standards for using hashing 
  algorithm and their bit size have been released by \href{"http://www.nist.gov/index.html"}
  {(NIST)} which are denoted by nomenclature Standard Hashing Algorithm (SHA).

  Due to advances in cryptanalysis of SHA-2, NIST announced a competition in November, 2007 
  to choose SHA-3. In October, 2012 the winner was selected to be \href{"http://keccak.noekeon.org/"}
  {Keccak} amongst 64 submissions. All the submissions were open to public scrutiny, and underwent
  intensive third party cryptanalysis, before the winner was selected. 
  \href{"http://csrc.nist.gov/groups/ST/hash/sha-3/sha-3\_selection\_announcement.pdf"}{Keccak was chosen}
   for its flexibility, efficient and elegant implementation, and large security margin.

  All algorithms submitted to competition have undergone public scrutiny. And other four finalist in 
  the competition were almost equivalent to Keccak, in attributes of security margin and implementation.
  In this project, I will be comparing Keccak with two other SHA-3 finalists, \href{"https://131002.net/blake/"}
  {BLAKE}, and \href{"http://www.groestl.info/"}{Gr$\o$stl} with respect to their resistance to 
  \href{"http://en.wikipedia.org/wiki/Simulated\_annealing"}{simulated annealing} and 
  \href{"http://en.wikipedia.org/wiki/Tabu\_search"}{tabu search}.
  
  Application of tabu search and simulated annealing to hash algorithms will be akin to generic attacks.
  That is these methods of breaking hash functions are design agnostic or do not depend on the workings
  of the hash function. Thus ensuring no bias in the experiment. At present, it is computationally infeasible
  to break the above mentioned hash functions. But the reduced versions of these can be subjected to attacks
  for near collisions. Thus I will be able to examine and conclude, if reduced instance Keccak has better 
  resistance to generic attacks than reduced instance of BLAKE and Gr$\o$stl.
}
\end{center}

\section{Problem Statement}

\subsection{Hash Functions}
A cryptographic hash function, is an algorithm capable of intaking arbitrarily long input string, and
output a fixed size string, often as called message digest. The message digest for two strings
even differing by a single bit should ideally be completely different, and no two input message should
have the same hash value. This property enables us to finger print a message. Following are the properties
of and ideal hash function. \cite{00005}
  
1. {\bf Preimage resistance}
\begin{center}
  \framebox
  {
    \parbox{350pt}
    {
      \centering \textsc{Preimage} \\
      {\bf Given:} A hash function $h : \mathcal{X} \to \mathcal{Y}$ and an element $y \in \mathcal{Y}$. \\
      {\bf Find:} $x \in \mathcal{X}$ such that $h(x) = y$. 
    }
  }
\end{center}
\vspace{4mm}
If the preimage problem for a hash function cannot be efficiently solved, then it is preimage resistant.
That is the hash function is one way, or rather it is difficult to find the input, given the output alone.

2. {\bf Second preimage resistance}
\begin{center}
  \framebox
  {
    \parbox{350pt}
    {
      \centering \textsc{Second preimage} \\
      {\bf Given:} A hash function $h : \mathcal{X} \to \mathcal{Y}$ and an element $x \in \mathcal{X}$. \\
      {\bf Find:} $x' \in \mathcal{X}$ such that $x' \neq x$ and $h(x) = h(x')$. 
    }
  }
\end{center}
\vspace{4mm}

A hash function for which a different input given another input, that compute to same hash cannot be found 
easily, is called as having second preimage resistance.

3. {\bf Collision resistance}
\begin{center}
  \framebox
  {
    \parbox{350pt}
    {
      \centering \textsc{Collision} \\
      {\bf Given:} A hash function $h : \mathcal{X} \to \mathcal{Y}$ 
      {\bf Find:} $x, x' \in \mathcal{X}$ such that $x' \neq x$ and $h(x') = h(x)$. 
    }
  }
\end{center}
\vspace{4mm}

Collision problem states that, can two different input strings be found, such that they hash to the same
 value given the same hash function. A hash function is collision resistant, if it is computationally
 infeasible to find two different values hashing to same value.

\subsection{Standards and NIST Competition}

Since hash functions can finger print any data, they find wide applications in computer security. And thus there
needs to be a standard for implemenation and application of hash function, which is provided by National Institute
 of Standards and Technology(NIST). 
 
\section{Background}

\section{Related Work}

\section{Methodology}

\section{Evaluation and expected outcomes}

\bibliographystyle{plain}
\bibliography{CapstoneBib}

\end{document}
