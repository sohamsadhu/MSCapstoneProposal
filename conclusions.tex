\chapter{Evaluation and expected outcomes}

%\section{Current Status}

%\section{Future Work}

%\section{Lessons Learned}

The experiment with each of the pair, will be run for approximately $2^{10}$ times or rather 1024 times.
The following are the parameters on which I propose to evaluate the algorithms. 

\begin{enumerate}
  \item How much time for each of the respective message digest size, did it take for the hill climbing
    algorithm to find a collision.
  \item How much is the hamming distance on an average for the chaining value that is manipulated by
    hill climbing for each of the algorithms.
  \item Till how many rounds, is the hill climbing algorithm a feasible option to find near collisions,
    for each of the algorithms.
  \item Is the weight of the hamming distance between message pair co-related to the amount of work
    hill climbing algorithm does to find a collision.
  \item Does the hamming weight distance between message pair, vary for each of the hashing algorithm
    with respect to amount of work on average required by hill climbing algorithm. This will be an
    indication, on how much diffusion each of the reduced versions of the algorithm are able to obtain.
  \item On an average what was the hamming distance of the chaining value obtained from a successful
    experiment from the individual message. This will help in understanding if, chaining values and
    message are closely related in reduced rounds.
\end{enumerate}

The above list is tentative, and by no means exhaustive. If during the course of experiment, more 
interesting figures come in front, then they will be added.
