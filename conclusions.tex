\chapter{Conclusions and future work}

%\section{Current Status}

%\section{Future Work}

%\section{Lessons Learned}

%The experiment with each of the pair, will be run for approximately $2^{10}$ times or rather 1024 times.
%The following are the parameters on which I propose to evaluate the algorithms. 

%\begin{enumerate}
  %\item How much time for each of the respective message digest size, did it take for the hill climbing
    %algorithm to find a collision.
  %\item How much is the hamming distance on an average for the chaining value that is manipulated by
    %hill climbing for each of the algorithms.
  %\item Till how many rounds, is the hill climbing algorithm a feasible option to find near collisions,
    %for each of the algorithms.
  %\item Is the weight of the hamming distance between message pair co-related to the amount of work
    %hill climbing algorithm does to find a collision.
  %\item Does the hamming weight distance between message pair, vary for each of the hashing algorithm
    %with respect to amount of work on average required by hill climbing algorithm. This will be an
    %indication, on how much diffusion each of the reduced versions of the algorithm are able to obtain.
  %\item On an average what was the hamming distance of the chaining value obtained from a successful
    %experiment from the individual message. This will help in understanding if, chaining values and
    %message are closely related in reduced rounds.
%\end{enumerate}

%The above list is tentative, and by no means exhaustive. If during the course of experiment, more 
%interesting figures come in front, then they will be added.

\section{Validity of the hypothesis}

In section 4.4 two hypothesis were suggested, both have been sort of some what disproved. When the reduced version
of an algorithm is only done by reducing the number of rounds, then for just rounds 1 and 2, Keccak performs worse
than BLAKE and Gr{\o}stl. Also from the numbers and taking into account the number of success in getting collisions,
hill climbing performs better than simulated annealing and tabu search. Random selection also performs, some what 
comparable to simulated annealing, but not much. Suggesting that for reduced rounds, that random selection could
also be used to find near collisions.

%Give the explanation later, first get the conclusions in.

\subsection{Collision resistance of SHA-3 finalist in reduced rounds}

For 1 round permutation, all the algorithms should almost equivalent weakness, that is collision in almost all instances
that is message pairs, and trial. However for Keccak for the three collision finding algorithms hill climbing, simulated
annealing, and random selection; collisions were seen in every instance and trial. This behaviour is again noticed
in Keccak for permuation of round 2; however for BLAKE and Gr{\o}stl, this number of trials and instances in which
collision could be found are reduced.

However as the permutation rounds are increased to 3 and 4, the resistance to collision finding algorithms for Keccak
becomes equivalent to that of BLAKE and Gr{\o}stl. For digest sizes, of 224 and 256; some message pairs do show
collision in trials less than 10. But for digest sizes of 384 and 512; no collisions are found.

\subsection{Feasibility of the collision algorithms}

From the collision algorithms; hill climbing seems to be the most optimised in terms of collision finding to time required
ratio. Tabu search performs the worst, in terms of collision to CPU time required. For example using tabu search on 
Gr{\o}stl for message digest of 224 bits, and rounds 1 and 2; tabu search required around 334894 iterations on average,
without any collisions. For the same conditions hill climbing did not require more than 900 iterations, with some success. 

\section{Effect of digest size}

\section{Effects of the number of rounds}

\section{Chaining value length}

\section{Bit differences in message in particular positions}

\section{Future work}

\begin{enumerate}
\item 
\end{enumerate}
