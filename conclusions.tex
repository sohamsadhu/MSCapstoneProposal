\chapter{Conclusions and future work}

\section{Validity of the hypothesis}

In section 4.4 three hypothesis were suggested, two have been disproved and one confirmed. When the reduced version
of an algorithm is only done by reducing the number of rounds, then for just rounds 1 and 2, Keccak performs worse
than BLAKE and Gr{\o}stl. Also from the numbers and taking into account the number of success in getting collisions,
hill climbing performs better than simulated annealing and tabu search. Random selection also performs, some what 
comparable to simulated annealing, but not much. Suggesting that for reduced rounds, that random selection could
also be used to find near collisions. The collision resistance of Keccak remains does not change, when the internal
state size of Keccak is reduced.

\subsection{Collision resistance of SHA-3 finalist in reduced rounds}

For 1 round permutation, all the algorithms should almost equivalent weakness, that is collision in almost all instances
that is message pairs, and trial. However for Keccak for the three collision finding algorithms hill climbing, simulated
annealing, and random selection; collisions were seen in every instance and trial. This behaviour is again noticed
in Keccak for permuation of round 2; however for BLAKE and Gr{\o}stl, this number of trials and instances in which
collision could be found are reduced.

However as the permutation rounds are increased to 3 and 4, the resistance to collision finding algorithms for Keccak
becomes equivalent to that of BLAKE and Gr{\o}stl. For digest sizes, of 224 and 256; some message pairs do show
collision in trials less than 10. But for digest sizes of 384 and 512; no collisions are found.

\subsection{Feasibility of the collision algorithms}

From the collision algorithms; hill climbing seems to be the most optimised in terms of collision finding to time required
ratio. Tabu search performs the worst, in terms of collision to CPU time required. For example using tabu search on 
Gr{\o}stl for message digest of 224 bits, and rounds 1 and 2; tabu search required around 334894 iterations on average,
without any collisions. For the same conditions hill climbing did not require more than 900 iterations, with some success.

At lower permutation rounds like 1 and 2, for Keccak; and for round 1 for BLAKE and Gr{\o}stl random selection is as 
effective as hill climbing and simulated annealing. Even for higher rounds like 3 and 4; for lower digest sizes of 224
and 256, random selection was able to find collision amongst all three SHA-3 candidate algorithms in around 10 instances 
in trials less than 10. This performance is comparable to that of simulated annealing.

Tabu search performs poorly, given that it goes for the steepest descent approach. Tabu search tries to pick the best 
chaining value from the neighbourhood, where the k-bit neighbourhood, whose size is roughly equal to $n^{2}$ where n
is the bit length of the chaining value. Also the tabu search has to go through elements of neighbourhood comparing each
of them with the tabu list, before inserting them into the candidate set where they will be considered as solution.
This places additional overhead on the algorithm performance. The tabu list can be made up of states, solutions previously
found, moves previously made. For our case the best element for tabu list would have been elements from previous iteration
that were not selected, but this still makes our set large. Although the process can be speed up, by getting the best
element from neighbourhood during neighbourhood creation, but you still have the over head of comparing the elements
twice for the best element and the comparing with tabu list. The problem does not mould well, with the tabu search
solution, because trying to store state, previous solution, or moves in tabu list are all expensive.

Simulated annealing though trying to avoid the local minima, does not perform any better than hill climbing, when finding
near collisions. Over trials over 256 and 128, and chaining values 32 and 64; simulated annealing found far fewer collisions
than hill climbing. While hill climbing was able to find collision in instances in more than 50\% of the message
pairs for each of the SHA-3 algorithm for message digest of 224 and 256, for permutation rounds of 3 and 4. But 
simulated annealing is only able to find collision in 1 instance out of 60 for 1 trial.

\subsection{Affects of state size for Keccak}

When Keccak is reduced to only 1 or 2 permutation rounds, then Keccak with lower state size, seems to possess better
collision resistance properties to hill climbing. As seen from table 6.11 and 6.26, for lower rounds the standard Keccak
version has comparatively weaker resistance, and requires least effort to find near collisions, for rounds 1 and 2.
In reduced state versions of Keccak, the message gets broken down into smaller blocks, and thus more bits are subjected
to more Keccak permutations as opposed to larger state size. Larger Keccak state size allows larger bit rates, thus
allowing less number of absorption and squeezing phases for message as opposed to smaller versions of Keccak. This
leads to message undergoing permutation rounds in reduced state size Keccak more times than that in full versions.

The collision resistance however plateaus for all versions of Keccak, after rounds are increased 3 and above. For Keccak
with digest sizes 224 and 256 bits, near collisions are still visible as seen from tables 6.27 to 6.30. However, when 
larger digest size of 384 and 512 are selected, then collisions are less visible. We can conclude from this, that
Keccak permutation rounds are good at diffusion, and is independent of state size it is acting upon.

\newpage

\section{Effect of digest size}

It seems as the digest size is increased, the near collision resistance increases, only if the rounds are above 2. The
input text here "The quick brown fox jumps over the lazy dog" was of 344 bits, and as per table 7.1 even with 32 bit
chaining value, the entire message is treated in one part itself, as seen from table 7.1. Thus our modifications to 
the message pairs do make it through in one go, rather than blocks of message from the pairs which have all bits same.

The reason higher message digests seem to have more collision resistance may be due to the fact that they have more
bits to shuffle around, thus making it harder for simple algorithms without any heuristics to find the collisions. Note
should be made that near collisions in this case is done when bits matching are more than 65\% of the digest length.
So it could be that bits upto 64\% might match and yet be rejected not as near collision found. Also the higher
digest sizes like 384 and 512 only show collision resistance only if applied with rounds 3 and 4.

Thus the minimum security recommendation for collision avoidance would be go for digest sizes of 384 and above.

\begin{table}
  \begin{center}
    \begin{tabular}{ | c | c | c | c | c | }                                 \hline
     \multirow{2}{*}{Hash algorithm} & \multicolumn{4}{ c|}{Digest size} \\ \cline{2-5}
               & 224  & 256  & 384  & 512  \\ \hline
     BLAKE     & 512  & 512  & 1024 & 1024 \\ \hline
     Gr{\o}stl & 512  & 512  & 1024 & 1024 \\ \hline
     Keccak    & 1152 & 1088 & 832  & 576  \\ \hline
    \end{tabular}
    \caption{Number of input bits to one function block, in the respective SHA-3 finalist algorithm}
  \end{center}
\end{table}

\begin{table}
  \begin{center}
    \begin{tabular}{ | c | c | c | c | c | }                                 \hline
     \multirow{2}{*}{Hash algorithm} & \multicolumn{4}{ c|}{Digest size} \\ \cline{2-5}
               & 224 & 256 & 384 & 512 \\ \hline
     BLAKE     & 14  & 14  & 16  & 16  \\ \hline
     Gr{\o}stl & 10  & 10  & 14  & 14  \\ \hline
     Keccak    & 24  & 24  & 24  & 24  \\ \hline
    \end{tabular}
    \caption{Number of permutation rounds, in the respective SHA-3 finalist algorithm}
  \end{center}
\end{table}

\newpage

\section{Effects of the number of rounds}

From the observation of tables 6.7 to 6.20, we can conclude that higher the number of permutation rounds, the
more is the collision resistance. Rounds 1 and 2, can be considered as not secure. It should be noted that
for permutation rounds 1 and 2; Keccak seems to perform poorly against BLAKE and Gr{\o}stl. But as rounds
are increased to 3 and 4, the performance of Keccak with respect to other algorithms is comparable. 

On the basis of observation it can be stated that, at minimum 3 rounds of permutation are required, and 4 to 
be secure as per the results of this experiment with limited resources. Please note that as per table 7.2,
the recommended number of permutation rounds for each SHA-3 finalist algorithm is different. While Gr{\o}stl
has the least number of rounds, Keccak has most number of rounds. This could be one explanation of Gr{\o}stl
and BLAKE's, better performance though not most ideal; than Keccak for rounds 1 and 2.

\section{Chaining value length}

Longer chaining value do not seem to help in getting near collisions. For perfect collisions, longer chaining values
might be helpful. However for near collisions, shorter chaining values work fine. With larger chaining values,
the number of times the algorithm has to loop through also increases. For increase in chaining value from 32 bits
to 64 bits, the loop execution increased by factor of 3 times more than what it required for 32 bits.

\section{Bit differences in message in particular positions}

All the SHA-3 algorithms displayed lack of diffusion property if any, irrespective of input bit updated in any
of the three positions start, middle and end specified in the message input string. The bit update made to middle,
and end to the input message are almost equivalent, since, the input message is padded with the chaining value. 
Thus we do not experiment with the bits updated at the end of the input string, only with bits updated at start
of the string or in the middle of the string. Since in the experiment trials the chaining value is chosen randomly,
hence small differences in finding collisions amongst the input, for a given hashing algorithm, digest size and round
can be ignored. Based on this fact, it seems that diffusion in the given algorithm is place agnostic. 

\section{Future work}

\begin{enumerate}
\item This experiment can be repeated with the two other finalist algorithm Skein and JH, alongwith Keccak.
\item Instead of reducing the number of rounds, the reduced version of the finalist algorithm could be obtained
by reducing the state size, and repeating this experiment.
\item Experiments can also be done with other combinations of factors of hashing algorithms, like reducing the
digest size, but increasing the number of rounds.
\item In this experiment, the near collision was benchmarked at 65\% of the output bits match. Instead further work
can be done to find the effort required to find perfect collisions for message pairs for reduced rounds for
certain hashing algorithms.
\end{enumerate}
