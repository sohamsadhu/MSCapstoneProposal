\chapter{Research Approach and Methodology}

%% Obviously you need to delete these lines when you have written up your text

%(Note: this chapter may be merged with Chapter 2 to have a combined Design 
%and Implementation chapter, if more appropriate.)

%\begin{itemize}
%\item{} Software details (use as many section as needed for class design, 
%database tables, middleware, etc.)

%\item{} Make sure you present and comment on any interesting issues about
%your implementation that you are proud of or unhappy with

%\item{} Skip code listing and specific UML diagrams, etc. to an appendix

%\end{itemize}

\section{Architecture}

Since, I plan on choosing Java as the primary programming language, hence the design will be 
object oriented based. The initial data that needs to be calculated for all the pairs of message
will be static for the rest of the experiment. Hence those can be obtained and stored for rest
of the experiment.

The first task will be creation of the data. First, pairs of initial message will have to be 
made. Each of the message from the pair will be line separated, and each of the pair in the file
will be separated with a blank line. This will be the initial data file. The hash function 
implementation of the algorithms already exist, so would be using them rather than coding them 
myself. This would avoid any bugs that could come due to immature understanding of how the compiler
handles large numbers. The implementations can then be tweaked to produce message digests with 
reduced rounds. These message digests will be different than the actual hash message obtained from
those functions since the rounds have been reduced.

The output for the results will be stored in the following format. The directory structure for the
output will be digest size. Followed by algorithm name whose digest pairs are being examined. Followed
by the file name for that particular message pair. For this message I intend to create 15 messages,
and they can be named from A to O. Thus a pairing of first message to second message will make the
output file name to be AB.txt. So when the hill climbing algorithm evaluates the hash values pairs
of Keccak algorithm with digest size of 512 bits, and for message pair. Then the output will be stored
in directory hierarchy as 512/Keccak/AB.txt.

The output file will be organised in same way as the input files, with each data line separated, and 
each experiment data separated by a blank line. The output for each experiment in hill climbing will
have the bit representation of the XOR value of two message digests, along with the chaining value
that was last obtained and the time taken for that experiment.

\section{Platform, Languages and Tools}

The platform I would most likely choose will be Ubuntu 12.04 LTS, with the primary coding language
being Java. The other minor book keeping tasks like generation of strings for message and chaining
value would be done with scripting language like Python. The choice of Ubuntu is based on fact, that
most machines as RIT CS department run on Ubuntu and hence, the experiments could be replicated
on that platform. Java with its wide range of packages, and execution speeds closer to C and C++,
would be an ideal choice to run repeated experiments. The file handling, time keeping, bit 
manipulation libraries that come with Java, also make it an ideal tool for such a mathematical 
intensive exercise.

\newpage

\section{Proposed schedule}

\begin{table}[h]
  \begin{center}
    \begin{tabular}{ | p{11.5cm} | c | } \hline
      Tasks                                                                                                   & Timeline \\ \hline
      Project proposal Approved. Completing 3rd party cryptanalysis part of report for Keccak.                & July 26 \\ \hline
      Creation of message pairs and coding 2 hash functions. Write cryptanalysis part for BLAKE in report.    & August 2 \\ \hline
      Coding the 3rd hash function, validation testing, finishing cryptanalysis part for Gr{\o}stl in report. & August 9 \\ \hline
      Code and test the hill climbing algorithm, and start collecting data from output.                       & August 16 \\ \hline
      Run experiments, collect more data, and put them in report. Discuss the results with advisor.           & August 23 \\ \hline
      Fine tune the experiment, collect data and start writing observations and conclusion part in report.    & August 30 \\ \hline
      Discuss results and conclusions with advisor. Fine tune report. Run more experiments if required.       & September 6 \\ \hline
      Format the report properly, and submit it for acceptance. Create presentation for project defense.      & September 13 \\ \hline
      Check availability of faculty. Announce defense date, book room and defend by September 20              & September 20 \\ \hline
    \end{tabular}
  \caption{Proposed schedule for my project implementation.}
  \end{center}
\end{table}

