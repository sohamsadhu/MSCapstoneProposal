\chapter{Observations, conclusions and future work}

%% Obviously you need to delete these lines when you have written up your text

%\begin{itemize}
%\item{}  How did you analyze your hyptothesis? Experiments, what did you think were worth measuring, etc.
%\item{}  Based on your measurements and qualititative analyses, how well did your approach work out?
%\item{}  Use graphs, tables, and other diagrams to illustrate your analyses.
%\item{}  Based on your analyses, how well does your implementation or approach match your hypothesis?
%\item{}  What do you deduce from this effort? How would you change or tweak your hypothesis? 

%\end{itemize}

%For the time being, here is my hypothesis, or the premise of my question. Keccak has been selected over BLAKE and
%Groestl for what? Is there is a basis that Keccak's property is still comparitively immune to zero sum distinguisher
%compared to BLAKE and Groestl in the reduced versions. This is what, I would like to find out and examine.

\section{Observations}
The experiment first subjected all the 3 SHA-3 finalist algorithm to hill climbing, simulated annealing, and
random selection; for permutation rounds of 1, 2, 3. The chaining value length was kept at 32 bits; and all
the possible four message digest sizes were evaluated. Tabu search algorithm was applied only to Gr{\o}stl
for rounds 1 and 2, for digest size of 224 bits. At first time, each collision finding algorithm was given
128 trials of experiment, for each hashing algorithm for a particular digest size and permutation round and
input class. The experiment, was again conducted as mentioned; but with chaining value of length 64 bits. Since
tabu search had previously proved to be expensive, with absence of encouraging results; it was discontinued.
The experiment on random selection was not done, since hill climbing and simulated annealing took much longer
time, than it took for the chaining value of 64 bits, and even that was discontinued. Since the results from
the 64 bit length chaining value weren't that encouraging, hence we stuck to chaining value of bit length 32.
We this time repeated our experiment, but with 256 trials instead of 128 trials, and experimented with rounds
of 3 and 4. All the following observations and numbers can be found on the 
\href{https://github.com/sxs9174/MSProjectCode/tree/master/MSProjectCode/Output}{git repository} for the source
code implementation of the experiment.

\subsection{Average iterations}

\begin{table}
  \begin{center}
    \begin{tabular}{ *{5}{| c |} }                               \hline
                 & \multicolumn{4}{|c|}{Rounds}               \\ \hline
     Digest Size & 1        & 2        & 3        & 4         \\ \hline
     224         & 803.6345 & 891.3258 & 886.0804 & 883.1557  \\ \hline
     256         & 804.5923 & 884.687  & 892.14   & 888.828   \\ \hline
     384         & 1137.889 & 898.263  & 899.4    & 902.6081  \\ \hline
     512         & 1118.947 & 902.052  & 903.687  & 906.0337  \\ \hline
    \end{tabular}
    \caption{Average iterations over all input cases for Hill Climbing for BLAKE}
  \end{center}
\end{table}

\subsection{Near collisions found}

\subsection{Time taken and outliers}

\section{Conclusions}

\subsection{Effect of digest size}

\subsection{Effects of the number of rounds}

\subsection{Chaining value length}

\subsection{Bit differences in message in particular positions}

\subsection{Feasibility of the collision algorithms}

\section{Future work}

\begin{enumerate}
\item
\end{enumerate}
