\chapter{Observations, conclusions and future work}

%% Obviously you need to delete these lines when you have written up your text

%\begin{itemize}
%\item{}  How did you analyze your hyptothesis? Experiments, what did you think were worth measuring, etc.
%\item{}  Based on your measurements and qualititative analyses, how well did your approach work out?
%\item{}  Use graphs, tables, and other diagrams to illustrate your analyses.
%\item{}  Based on your analyses, how well does your implementation or approach match your hypothesis?
%\item{}  What do you deduce from this effort? How would you change or tweak your hypothesis? 

%\end{itemize}

%For the time being, here is my hypothesis, or the premise of my question. Keccak has been selected over BLAKE and
%Groestl for what? Is there is a basis that Keccak's property is still comparitively immune to zero sum distinguisher
%compared to BLAKE and Groestl in the reduced versions. This is what, I would like to find out and examine.

\section{Observations}
The experiment first subjected all the 3 SHA-3 finalist algorithm to hill climbing, simulated annealing, and
random selection; for permutation rounds of 1, 2, 3. The chaining value length was kept at 32 bits; and all
the possible four message digest sizes were evaluated. Tabu search algorithm was applied only to Gr{\o}stl
for rounds 1 and 2, for digest size of 224 bits. At first time, each collision finding algorithm was given
128 trials of experiment, for each hashing algorithm for a particular digest size and permutation round and
input class. The experiment, was again conducted as mentioned; but with chaining value of length 64 bits. Since
tabu search had previously proved to be expensive, with absence of encouraging results; it was discontinued.
The experiment on random selection was not done, since hill climbing and simulated annealing took much longer
time, than it took for the chaining value of 64 bits, and even that was discontinued. Since the results from
the 64 bit length chaining value weren't that encouraging, hence we stuck to chaining value of bit length 32.
We this time repeated our experiment, but with 256 trials instead of 128 trials, and experimented with rounds
of 3 and 4. All the following observations and numbers can be found on the 
\href{https://github.com/sxs9174/MSProjectCode/tree/master/MSProjectCode/Output}{git repository} for the source
code implementation of the experiment.

\subsection{Choice of programming language}

Java comes as not the best choice in implementing algorithms related to hashing. Although since this experiment 
required GUI, and platform compatibility which made Java ideal tool. The SHA-3 finalist algorithms have word size
or state size parameters of 32 or 64 bits, which exploit the computer architecture of having these word sizes
built in or used, and using them as single data entities. In languages like C, there are primitive data types
of size 4 bytes or 8 bytes, which can be easily integrated into the hashing algorithm. Java also has similar
word size, but by default the data type is always signed. This creates problems while bit shifting or implementing
some of the modulus operations. The leading bit is zero and during bit shifts if the bit is toggled to 1, then 
the data in it will treated as negative number, due to which some of the optimizations cannot be performed easily.

\subsection{Average iterations}

Table 6.1 to 6.3 show the average iterations for the hill climbing algorithm applied to the 3 SHA-3 finalist algorithm,
for a chaining value of length 32 bits. For most of the algorithms except for the BLAKE for 1 permutation round
the number of iterations increase as the digest size is increased. Also the number of iterations also seem to
increase when number of rounds are increased, but at round 3 and 4 they are almost similar. For 32 bit chaining value
the number of iterations for hill climbing algorithm applied to all SHA-3 finalist algorithm, remains around 900.
For Keccak in hill climbing, when the permutation round was set at 1, the average iterations it took was around 532,
lower than other SHA-3 finalist for same number of rounds. The number of iterations for random selection, simulated
annealing for 32 bit chaining value were fixed at 1024 iterations.

The average iterations for tabu search for Gr{\o}stl message for digest length 224 and permutation round 1 and 2
was 335254.443. Also there were no collisions found with this method. Being so computationally expensive and with
no positive results, the experiment with tabu search was discontinued.

As the number of bits in chaining value is increased from 32 to 64, the number of iterations increases to around 3500.
The increase in number of iterations is more than three and half times, than what is observed in 32 bit chaining value.
\begin{table}
  \begin{center}
    \begin{tabular}{ *{5}{| c |} }                                 \hline
                 & \multicolumn{4}{|c|}{Rounds}                 \\ \hline
     Digest Size & 1         & 2         & 3         & 4        \\ \hline
     224         & 803.6331  & 891.3276  & 886.56465 & 883.1557 \\ \hline
     256         & 804.5944  & 891.35455 & 892.13813 & 885.4943 \\ \hline
     384         & 1137.8868 & 898.2596  & 899.3996  & 902.6081 \\ \hline
     512         & 1118.9453 & 902.0659  & 903.68877 & 906.0339 \\ \hline
    \end{tabular}
    \caption{Average iterations over all input cases for Hill Climbing for BLAKE for chaining value
    of bit length 32}
  \end{center}
\end{table}

\begin{table}
  \begin{center}
    \begin{tabular}{ *{5}{| c |} }                                  \hline
                 & \multicolumn{4}{|c|}{Rounds}                  \\ \hline
     Digest Size & 1         & 2         & 3         & 4         \\ \hline
     224         & 875.893   & 888.3345  & 886.71898 & 885.8383  \\ \hline
     256         & 889.72437 & 887.639   & 891.2398  & 889.7382  \\ \hline
     384         & 872.1038  & 897.5908  & 898.66531 & 897.1204  \\ \hline
     512         & 896.17804 & 904.18335 & 904.1414  & 902.55646 \\ \hline
    \end{tabular}
    \caption{Average iterations over all input cases for Hill Climbing for Gr{\o}stl for chaining value
    of bit length 32}
  \end{center}
\end{table}

\begin{table}
  \begin{center}
    \begin{tabular}{ *{5}{| c |} }                                 \hline
                 & \multicolumn{4}{|c|}{Rounds}                 \\ \hline
     Digest Size & 1         & 2         & 3         & 4        \\ \hline
     224         & 532.2517  & 880.6999  & 883.23035 & 888.5428 \\ \hline
     256         & 532.341   & 888.98126 & 889.24855 & 895.0305 \\ \hline
     384         & 533.9329  & 892.61993 & 899.02125 & 904.8013 \\ \hline
     512         & 533.70807 & 900.46466 & 905.87    & 902.2507 \\ \hline
    \end{tabular}
    \caption{Average iterations over all input cases for Hill Climbing for Keccak for chaining value
    of bit length 32}
  \end{center}
\end{table}

\begin{table}
  \begin{center}
    \begin{tabular}{ *{4}{| c |} }                      \hline
                 & \multicolumn{3}{|c|}{Rounds}      \\ \hline
     Digest Size & 1         & 2         & 3         \\ \hline
     224         & 4190.473  & 3465.0413 & 3483.752  \\ \hline
     256         & 4264.3496 & 3456.9246 & 3431.7415 \\ \hline
     384         & 3885.6829 & 3515.1746 & 3528.4922 \\ \hline
     512         & 3984.2366 & 3535.423  & 3559.4246 \\ \hline
    \end{tabular}
    \caption{Average iterations over all input cases for Hill Climbing for BLAKE for chaining value
    of bit length 64}
  \end{center}
\end{table}

\begin{table}
  \begin{center}
    \begin{tabular}{ *{4}{| c |} }                      \hline
                 & \multicolumn{3}{|c|}{Rounds}      \\ \hline
     Digest Size & 1         & 2         & 3         \\ \hline
     224         & 3687.7493 & 3468.3171 & 3469.366  \\ \hline
     256         & 3714.0242 & 3479.8074 & 3473.6194 \\ \hline
     384         & 3594.1716 & 3522.1423 & 3512.8254 \\ \hline
     512         & 3581.0823 & 3544.302  & 3559.9834 \\ \hline
    \end{tabular}
    \caption{Average iterations over all input cases for Hill Climbing for Gr{\o}stl for chaining value
    of bit length 64}
  \end{center}
\end{table}

\begin{table}
  \begin{center}
    \begin{tabular}{ *{4}{| c |} }                      \hline
                 & \multicolumn{3}{|c|}{Rounds}      \\ \hline
     Digest Size & 1         & 2         & 3         \\ \hline
     224         & 2118.8818 & 3535.5684 & 3457.5586 \\ \hline
     256         & 2118.3499 & 3557.9949 & 3466.1624 \\ \hline
     384         & 2134.1736 & 3676.9707 & 3521.3984 \\ \hline
     512         & 2139.1523 & 3764.485  & 3562.9753 \\ \hline
    \end{tabular}
    \caption{Average iterations over all input cases for Hill Climbing for Keccak for chaining value
    of bit length 64}
  \end{center}
\end{table}

\subsection{Near collisions found}

\begin{table}
  \begin{center}
    \begin{tabular}{ *{10}{| c |} }                      \hline
                 & \multicolumn{9}{|c|}{Rounds}      \\ \hline
     Digest Size & \multicolumn{3}{|c|}{1} & \multicolumn{3}{|c|}{2} & \multicolumn{3}{|c|}{3} \\ \hline
                 & S & M & E               & S & M & E               & S & M & E               \\ \hline
     224         & S & M & E               & S & M & E               & S & M & E               \\ \hline
     256         & S & M & E               & S & M & E               & S & M & E               \\ \hline
     384         & S & M & E               & S & M & E               & S & M & E               \\ \hline
     512         & S & M & E               & S & M & E               & S & M & E               \\ \hline
    \end{tabular}
    \caption{Average iterations over all input cases for Hill Climbing for Keccak for chaining value
    of bit length 64}
  \end{center}
\end{table}


\section{Conclusions}

\subsection{Effect of digest size}

\subsection{Effects of the number of rounds}

\subsection{Chaining value length}

\subsection{Bit differences in message in particular positions}

\subsection{Feasibility of the collision algorithms}

\section{Future work}

\begin{enumerate}
\item
\end{enumerate}
