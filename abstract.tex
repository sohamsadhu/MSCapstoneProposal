\begin{abstractpage}

Digital signatures used in \href{"http://en.wikipedia.org/wiki/Cryptographic\_hash\_function\#Applications"}
{computer security}, are created and verified using cryptographic hash functions \cite{00042}.
Standard Hashing Algorithm(SHA) is the nomenclature for hashing functions that have been standardized
by \href{"http://www.nist.gov/index.html"}{National Institute of Standards and Technology(NIST)}.

Due to advances in cryptanalysis of SHA-2, NIST announced a competition in November, 2007 to choose SHA-3. 
All the 64 submissions to NIST were open to public for cryptanalysis. 
In October, 2012 \href{"http://keccak.noekeon.org/"}{Keccak} was announced as the winner for SHA-3 competition. 
\href{"http://csrc.nist.gov/groups/ST/hash/sha-3/sha-3\_selection\_announcement.pdf"}{Keccak was chosen}
for its robustness and high security margin.

Of the 64 submissions to SHA-3 competition, 56 were selected for the first round, and 5 made it final round.
In this project I have compared the resistance to near collisions, on reduced versions of Keccak and two other
finalist \href{"https://131002.net/blake/"}{BLAKE}, and \href{"http://www.groestl.info/"}{Gr{\o}stl}. 
A $\epsilon / n $ bit near collision for hash function h and two messages $M_{1}$ and $M_{2}$, where
$M_{1} \neq M_{2}$ is defined as
\begin{center}$HW( h( M_{1}, CV ) \oplus h( M_{2}, CV ) ) = n - \epsilon $\end{center}
where HW is the Hamming weight, h is the hash function, and CV is the chaining value, and n is the hash size 
in bits. I have also compared the near collision resistance amongst different versions of Keccak based on 
reduced internal state size.

Hill climbing, simulated annealing, tabu search and random selection are applied to find a neighbourhood of chaining
value that is appended to 2 messages that are dissimilar by a bit, to find near collisions in the reduced versions of
mentioned hashing algorithms.
  
\end{abstractpage}
