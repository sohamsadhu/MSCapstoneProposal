\chapter{Related work}

%From the wolfram alpha pages what is a field. A field is one that have field axioms of associativity, distributivity,
%commutativity, inverse, identity(else you cannot have the inverse). A field with finite elements or field order is finite
%or Galois field. GF(p) where the Galois field of order p. The order of the field is a power of a prime number. A GF 
%consists of residue classes of modulo p. Now a residue can be a congruence b mod n, then b is the residue. A finite
%field will have limited number of residues, which will form a residue class. The residue classes of a function x is
%all possible values of residue of f(x)(mod n). Galois fields are made of residues of the modulus function, so the 
%equivalence is based on the modulus function.

%OK, another thing, that I learned about in past few days was about Galois field. Why Galois field? Well, the thing
%is they are the building blocks to what is there in the cryptographic function. What did I learn about field, that
%number of elements in field are limited for the modulo of the prime that is the order of the field. Since it is
%modulo, so all the elements repeat with the numbers. The elements in a field obey the axioms of field that include
%associativity, distributive, commutative, inverse and identity. The modulo prime can be represented as a polynomial
%of odd powers summing to the power of the prime power of the field. The polynomial has to be irreducible, since if
%you allow reducible polynomial there is a possibility, that the polynomials would sum to the modulo and become a zero
%element that cannot be allowed to happen. Since multiplication with zero will be zero. Other than that, figuring
%out a inverse in field is hard but if you have the look up tables of logarithms with the generator numbers whose 
%successive powers modulo the prime generates all the numbers in the field. This table is then made as a look up,
%when you multiply the polynomials. There is a already algorithm and code written up for that thing. Which can be used.

\section{Cryptanalysis done on Keccak}

\subsection{Rotational cryptanalysis of round-reduced KECCAK}

%\textit{Rotational cryptanalysis\cite{00041} is used to construct a 5-round distinguisher on Keccak-f[1600]. And use
%it to get preimage upto 3 rounds for Keccak-f[r=1024, c=576], and a modified Keccak(without the iota function) upto
%4 rounds, with complexity of roughly 2\textsuperscript{506} calls to KECCAK-512.}

Rotational cryptanalysis\cite{00041} is used to to follow relation between states $(A, A^{\leftarrow})$ of 
KECCAK-f[1600] in course of their transformation, and thus derive a distinguisher.

\begin{defn}
A pair of two 1600-bit states $(A, A^{\leftarrow})$ is called rotational pair when each lane in state $A^{\leftarrow}$
is created by bitwise rotation of operation of corresponding lane in state $A$. The operation moves the bit from 
position $(x, y, z)$ to the position $(x, y, z + n)$, where z + n is done on modulo 64. $x$ and $y$ values range from 0 
to 4, and $z$ value ranges from 0 to 63. $n$ is the rotational number and is same for every lane. Thus rotational pairs
are $\forall(x, y, z) : A_{(x, y, z)} = A^{\leftarrow}_{(x, y, z + n)}$. \cite{00022}
\end{defn}

\begin{defn}
Set $S_n$ is a set of $2^{1600}$ pairs of states which are created by an operation of KECCAK-f[1600] applied to all
possible rotational pairs. \cite{00022}
\end{defn}

\begin{defn}
Probability $p^{n}_{(x, y, z)}$ is the probability for pair of states $(A, A^{\leftarrow})$ randomly selected from the
set $S_n$ we have $A_{(x, y, z)} = A^{\leftarrow}_{(x, y, z + n)}$. \cite{00022}
\end{defn}

\begin{defn}
Given probability distribution $\mathcal{D}_n$ that assigns probability $\frac{1}{n!}$ for each $p \in \mathcal{P}_n$.
A permutation is called random if it is chosen according to uniform distribution $\mathcal{D}_n$.\cite{00022}
\end{defn}

It is assumed that random permutation $p^n_{(x, y, z)}$ follows binomial distribution $\mathcal{B}(t, s)$ where $t$ is
trials and $s$ is success probability that is equal to 0.5. Experimental results for a chosen $p_{(x, y, z)}$ are 
compared to follow distribution $\mathcal{B}(t, s)$. The experimental values are supposed to fall within range of 
$0.5t\pm2\sigma$ with 95\% confidence interval.

The probability change through steps of Keccak, can be derived from analysis of bitwise operation. It is assumed
that corresponding bits from $(A, A^{\leftarrow})$ are equal, both combinations ('00' or '11') or opposite combinations
have same probability to be actual values. Operations like NOT, or rotation in Keccak do not affect the probabilities,
so only probabilities for AND and XOR are considered.

\begin{lem}
Given input bits a, b and output bit out; with $p_a$ and $p_b$ defined as per definition 4.1.1.2, then for AND operation
$P_{out} = \frac{1}{2}(p_{a} + p_{b} - p_{a} p_{b})$ \cite{00022}
\end{lem}

\begin{lem}
Given input bits a, b and output bit out; with $p_a$ and $p_b$ defined as per definition 4.1.1.2, then for XOR operation
$P_{out} = p_{a} + p_{b} - 2 p_{a} p_{b}$ \cite{00022}
\end{lem}

A 4 round rotational is built, and found that some values deviate from the 0.5 like $p^{54}_{(4, 4, 14)} = 0.5625$. 10,000
random samples of rotational pairs are ran on 4 round KECCAK-f[1600]. The mean from that sample is 5682 which is beyond
range from mean of $\mathcal{B}(10000, 0.5)$ which is 5000$\pm$2.5. Thus demonstrating a distinguisher for 4 rounds.
After 4 rounds all $p^n_{(x, y, z)} = 0.5$, and hence the distinguisher cannot be directly extended. But instead probability
that relation between two pairs of states $(A_{(x, y, z)}, A^{\leftarrow}_{(x, y, z+n)})$ and 
$(A_{(x, y', z)}, A^{\leftarrow}_{(x, y'', z+n)})$ are observed, that should follow distribution $\mathcal{B}(10000, 0.5)$.
Values for $p^63_{(2, 1, 37)}$ and $p^63_{(2, 2, 37)}$ have the highest deviation from 0.5 at end of fourth round. The
probability that they are in the same relation is given by $p^n_{(x, y, z)} \dot p^n_{(x, y'', z)} + (1 - p^n_{(x, y, z)})
(1 - p^n_{(x, y, z)})$ which comes to 0.499024. The $\rho$ and $\pi$ steps in Keccak only change the position of the
probability to bit pairs $(A_{(1, 2, 43)}, A^{\leftarrow}_{(1, 2, 44)})$ and $(A_(2, 0, 16), A^{\leftarrow}_{(2, 0, 17)})$.
The bias is observed by generating sufficient number of samples 'm' based on Chernoff bound based on inequality 
\[ m \geq \frac{1}{(P_c - 0.5)^2} \ln \frac{1}{\sqrt \in}\]
where $\in$ is error set to 0.05. m turns out to be 403,000,000. The following steps are implemented.\cite{00022}
\begin{enumerate}
\item 403,000,000 rotational pairs are generated.
\item For each pair
  \begin{enumerate}
  \item Run Keccak for 5 rounds on states $A$ and $A^{\leftarrow}$.
  \item if $(A_{(1, 2, 43)} \oplus A^{\leftarrow}_{(1, 2, 44)} \oplus A_{(2, 0, 16)} \oplus A^{\leftarrow}_{(2, 0, 17)} = 0)$
  then \newline mean := mean + 1
  \end{enumerate}
\end{enumerate}

The mean for $\mathcal{B}(403,000,000, 0.5)$ with the standard deviation has range of 201,500,000$\pm$2.10037, but 
experimentally from above procedure it comes to around 201,450,503, thus concluding this as a distinguisher.

For the preimage attack an unknown message with cyclical pattern like that of 4 0's followed by 4 1's alternatively of
512 bits is chosen. There are 256 possible messages of the cyclic pattern discussed. For the preimage attack, a rotational
counterpart of the preimage is searched for, that would reduce the complexity from random search. Following are the steps
\cite{00022}

\begin{enumerate}
\item Guess first 8 lanes of $A^{\leftarrow}$
\item Run Keccak-f[1600] for 3 rounds on state $A^{\leftarrow}$
\item for n:= 0 to n < 64 do
  \begin{enumerate}
  \item candidate := true
  \item 10 sets of cordinates $(x, y, z)$ being on list created on precomputation do
  \newline $(p^n_{(x, y, z)} = 0)$ and $(A_{(x, y, z) \neq A^{\leftarrow}_{(x, y, z)}})$ then candidate := false
  \newline $(p^n_{(x, y, z)} = 1)$ and $(A_{(x, y, z) = A^{\leftarrow}_{(x, y, z)}})$ then candidate := fals
  \item if (candidate = true) then rotate the guessed state by n bits. Verify by input to Keccak function, that
  runs for 3 rounds.
  \end{enumerate}
\end{enumerate}

\subsection{New Attacks on Keccak 224 and Keccak 256} \cite{00023}
The authors of this paper present practical-time collisions on Keccak[r=1088,c=512] (and lower capacity) with 4 rounds. 
They combine a low-weight trail over 3 rounds with algebraic techniques.

\subsection{Practical Analysis of Reduced Round Keccak} \cite{00024}
In this paper, the authors propose several practical-time attacks on the Keccak hash function with 2 to 4 rounds. 
First, they give a differential distinguisher exploiting a low-weight differential trail. Its complexity is 225 for 4 
rounds. Then, they show how to produce a collision (resp. near-collision) on 2 (resp. 3) rounds of Keccak[r=1088,c=512] 
(and lower capacity) with complexity 233 (resp. 225). Finally, they present an algorithm to find (second) preimages in 
time 231 and memory 229.

\subsection{Unaligned Rebound Attack: Application to Keccak} \cite{00025}
This paper analyzes two aspects of differential cryptanalysis on Keccak: efficient trails and rebound attacks. In the 
former, the authors propose a heuristic to build differential trails with a low restriction weight. For Keccak-f[1600], 
they obtained trails of weight 32, 142 and 709 for 3, 4 and 5 rounds, respectively. In the latter, the paper presents 
distinguishers making use of the rebound attack for up to 8 rounds of Keccak-f[1600] with a complexity of 2491.

\subsection{Improved zero-sum distinguisher for full round Keccak-f permutation} \cite{00026}
The authors of this paper noted a property of the inverse of the non-linear function χ: while χ-1 has algebraic degree 3,
the product of any two output bits also has degree 3. This allows to estimate the degree of the Keccak-f rounds more tightly 
and to extend the zero-sum distinguisher on Keccak-f[1600] to size 21575 for 24 rounds.

\subsection{A SAT-based preimage analysis of reduced Keccak hash functions} \cite{00027}
In this paper, Paweł Morawiecki and Marian Srebrny report on experiments for generating preimages using SAT solvers. 
They attack Keccak versions calling Keccak-f with width 50, 200 and 1600 and with a reduced number of rounds. 
They compare the SAT solver approach with plain exhaustive search and it turns out to be faster for up to 3 rounds.

\subsection{Zero-sum Distinguishers for Iterated Permutations and Application to Keccak-f and Hamsi-256} \cite{00028}
In this paper, Christina Boura and Anne Canteaut extend their zero-sum distinguishers to 20 rounds.

\section{Cryptanalysis done on BLAKE}

\section{Cryptanalysis done on Gr$\o$stl}
