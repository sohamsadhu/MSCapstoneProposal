\chapter{Background}

\section{Hashing}
A cryptographic hash function, is a function that can take string data of arbitrary length as input. 
And output a bit string of fixed length, that is ideally unique to the input string given. The 
aforementioned is description of a single fixed hash function. But, hash functions can be tweaked
with an extra key parameter. This gives rise multiple hash functions or \emph{hash family} as 
defined below \cite{00005}

\begin{center}
  \framebox
  {
    \parbox{420pt}
    {
      A \emph{hash family} is a four-tuple ($\mathcal{X}, \mathcal{Y}, \mathcal{K}, \mathcal{H}$),
      satisfying the following conditions.
      \begin{itemize}
        \item $\mathcal{X}$ is a set of possible messages
        \item $\mathcal{Y}$ is a finite set of hash function output
        \item $\mathcal{K}$, the \emph{keyspace}, is a finite set of possible keys
        \item For each $K \in \mathcal{K}$, there is a hash function $h_{k} \in \mathcal{H}$. Each 
          $h_{k}: \mathcal{X} \to \mathcal{Y}$ 
      \end{itemize}
    }
  }
\end{center}
\vspace{4mm}

In the above definition, $\mathcal{X}$ could be finite or infinite set, but $\mathcal{Y}$ is always
a finite set, since the length of bit string or hash function output, that defines $\mathcal{Y}$ is
finite. A pair (x, y) $\in \mathcal{X} \times \mathcal{Y}$ is a \emph{valid pair} under key K, if 
$h_{k}(x) = y$.

If $\mathcal{F}^{\mathcal{X}\mathcal{Y}}$ denotes set of all functions that map from domain $\mathcal{X}$
to co-domain $\mathcal{Y}$. And if $\mid\mathcal{X}\mid$ = N and $\mid\mathcal{Y}\mid$ = M, then 
$\mid\mathcal{F}^{\mathcal{XY}}\mid$ = $M^{N}$. Then any hash family $\mathcal{F} \subseteq \mathcal{F}^{\mathcal{XY}}$
is called as (N, M) - hash family.

An \emph{unkeyed hash function} is a function $h_{k}: \mathcal{X} \to \mathcal{Y}$, where $\mathcal{X}$ and
$\mathcal{Y}$ are as defined above, and where $\mid\mathcal{K}\mid$ = 1. Thus a single fixed function h(x) = y,
or an unkeyed hash function as hash family with only one key. For the purpose of this document, we will
be concentrating on unkeyed hash family or fixed hash functions only, and will be referring to them as
hash functions, unless mentioned otherwise.

The output of a hash function is generally called as a message digest. Since, it can viewed as a unique
snapshot of the message, that cannot be replicated if the bits in message are tampered with.
  
\subsection{Properties of an ideal hash function}
An ideal hash function should be easy to evaluate in practice. However, it should satisfy the following
three properties primarily, for a hash function to be considered \emph{secure}.

1. {\bf Preimage resistance}
\begin{center}
  \framebox
  {
    \parbox{300pt}
    {
      \centering \textsc{Preimage} \\
      {\bf Given:} A hash function $h : \mathcal{X} \to \mathcal{Y}$ and an element $y \in \mathcal{Y}$. \\
      {\bf Find:} $x \in \mathcal{X}$ such that $h(x) = y$. 
    }
  }
\end{center}
\vspace{4mm}

The problem preimage suggests that can we find an input $x \in \mathcal{X}$, given we have the hash 
output $y$, such that $h(x) = y$. If the preimage problem for a hash function cannot be efficiently
solved, then it is preimage resistant. That is the hash function is one way, or rather it is difficult
to find the input, given the output alone.

2. {\bf Second preimage resistance}
\begin{center}
  \framebox
  {
    \parbox{300pt}
    {
      \centering \textsc{Second preimage} \\
      {\bf Given:} A hash function $h : \mathcal{X} \to \mathcal{Y}$ and an element $x \in \mathcal{X}$. \\
      {\bf Find:} $x' \in \mathcal{X}$ such that $x' \neq x$ and $h(x) = h(x')$. 
    }
  }
\end{center}
\vspace{4mm}

Second preimage problem suggests that given an input $x$, can another input $x'$ be found, such that
$ x \neq x'$ and hash output of both the inputs are same, that is $h(x) = h(x')$. A hash function for
which a different input given another input, that compute to same hash cannot be found easily, is 
called as having second preimage resistance.

3. {\bf Collision resistance}
\begin{center}
  \framebox
  {
    \parbox{300pt}
    {
      \centering \textsc{Collision} \\
      {\bf Given:} A hash function $h : \mathcal{X} \to \mathcal{Y}$ \\
      {\bf Find:} $x, x' \in \mathcal{X}$ such that $x' \neq x$ and $h(x') = h(x)$. 
    }
  }
\end{center}
\vspace{4mm}

Collision problem states that, can two different input strings be found, such that they hash to the
same value given the same hash function. If the collision problem for the hash function, is computationally
complex, then the hash function is said to be collision resistant.

Basically, the above properties make sure that hash function has one to one mapping from input to
output, and is one way. That is if a two different input strings with even minute differences should
map to two different hash values. And it should be practically infeasible, to find a input given a
hash value. \\

\section{Security Model}

On the basis of above properties described for a hash function. A generic model of security fulfillment, 
for any hash function to be considered secure can be set. Two such ideas, that a hash function should
comply as far as possible to be considered as a hash function are described below.

  \subsection{ Random Oracle }
  Hash functions being built on mathematical operations, cannot be truly random, but are efficient
  approximations of fixed random output mapping to an input. An ideal hash function can be abstracted 
  as a random oracle, and the proofs can be formalized. To show that algorithm is secure modulo the way
  it creates random outputs \cite{00018}.

  Random oracle model, proposed by Bellare and Rogaway, is a mathematical model of ideal hash function.
  It can be thought of this way, that the only way to know the hash value for an input $x$ would be to
  ask the Oracle or rather compute the hash of the input itself. There is no way of formulating or 
  guessing the hash value for input, even if you are provided with substantial number of input and output
  pairs. It is analogous to looking up for corresponding value of the key in a large table. To know the
  value for an input, you look into the table. A well designed hash function mimics the behaviour as 
  close as possible to a random oracle. 

  \subsection{ Birthday Paradox }
  If we randomly choose 23 people, then the probability that two people from the group will have identical
  birthday is around 50\%. This is because, the first person can be paired with rest of 22 people in group,
  to form 22 pairs. The next person in group can be paired with remaining 21 people to get 21 pairs. Thus 
  we end up with 22 + 21 + 20 + \ldots + 1 = 253 pairs. Thus the probability is ratio of pairs 253 to the
  sample space 365 days in a year (ignoring the leap year).

  Two people with same birthday can be seen analogous to two inputs hashing to the same value, that is 
  collision. Say the sample space of hash as M, and denote the number of samples to be taken as N. Then 
  by birthday problem described above, the minimum number of people required (N) to have the same birthday 
  within a year (M = 365) with probability 0.5, would be N = 23.

  It can be formally proved for any sample size M, to find two values that are identical with probability
  0.5 can be given by the equation $N \approx 1.17 \sqrt{M}$. This can be interpreted as hashing over $\sqrt{M}$
  values roughly will give us two entries with 50\% probability of a collision.

  The above theorem can be applied following way. If we brute force to find collision in a hash function
  that has a message digest length of $2^{128}$ bits, then at minimum we would need to calculate $2^{64}$
  instances of hash, to find a collision with a probability of 50\%. Any good hash function in practice 
  should be resistant to attacks, that require operations less than that predicted by the birthday attack
  for that hash.

\section{Applications}

Applications of cryptographic hash functions, can be broadly classified in areas of verification, data
integrity and pseudo random generator functions.

  \subsection{Verification and data integrity}
  \begin{enumerate}

    \item Digital Forensics: When digital data is seized and to be used as evidence, a hash of the original
    digital media is taken. A copy of the digital evidence is made under the regulations, and the hash of the
    copied digital media is made, before it can be examined. After the evidence has been examined, then another
    hash value of the copy of the evidence that was used in examination is made. This ensures, that evidence
    has not been tampered. \cite{00013}

    \item Password verification: Passwords are stored as hash value, of password concatenated with some salt
    string. The choice of salt depends on implementation. When a password is to be verified, it is first 
    concatenated with the respective salt. A hash value of this new modified password string is taken and compared
    with the value stored in the database. If the values match, then the password is authenticated.

    \item Integrity of files: Hash values can be used to check, that data files have not been modified over the
    time in any way. Hash value of the data file taken at a previous time is checked with the hash value of the
    file taken at present. If the values do not match, it means that file in question has been modified over the
    time period between, when hash value of the file was taken and present.
  \end{enumerate}

  \subsection{Pseudo random generator function:}
  Cryptographic hash functions can be used as pseudo random bit generators. The hash function is initialised
  with a random seed, and then hash function is queried iteratively to get a sequence of bits, which look random.
  Since, the cryptographic hash algorithm is a mathematical function, so the sequence of two pseudo random bits 
  would be similar if they come from same hash function with the same key. And they would not be perfectly random.
